\documentclass[12pt, twoside]{amsart}    

\usepackage{../current_style}

\begin{document}

\title{Veggie Chili}
\maketitle
\subtitle{3-4 hours}

\noindent \desc{This is a fun recipe. It takes a while as the soup simmers for a few hours and then it takes an hour just to get to making the soup, but after the initial bit of making the soup, you can kind of leave it alone. If you make cornbread you also have to do that, but again you'll have a bit of downtime, and when there's another person in the kitchen, it's easy to tag-team with one person on the soup and the other on the cornbread. It also tastes good as leftovers. For beans I like black beans and pinto beans, but it is hard to go wrong as long as you have the right spices.}

\recipe{
\textbf{Soup:}
\begin{ingredients}
\item Soup Base
\item mostly cooked 1 cup dried bean of choice
\item 1 large can of tomatoes
\item Tomato Paste
\item 1 more teaspoon of coriander
\item 1 tbsp of cumin
\item Choice of Broccoli, Spinach, \\Squash, Kale, Green beans, \\finely chopped Cabbage. *
\item Chili in Adobo or smoked paprika to taste \\
\end{ingredients}
\textbf{Toppings}:
\begin{ingredients}
\item Sour Cream
\item Cheese
\item Cornbread (see recipe)
\end{ingredients}
}{
\begin{directions}
\item Follow the stew base recipe as written but remove 2-3 cups of liquid to make room for the tomatoes. Add up to and including the cumin with the liquid. 
\item While the soup simmers, make your favorite cornbread recipe
\item Ten minutes before you serve add whatever vegetables you want to steam at the end. Then season with salt and chili in adobo or smoked paprika and serve with toppings of your choice.
\end{directions}
\note{The bulk veggies are things which can cook in a reasonable amount of time (10 or so minutes at boiling) to add more stuff to the soup but retaining a distinct shape. This is particularly good with things like brocolli, green beans, or pre-boiled potatoes as they will mostly retain their taste as well giving some variety to the otherwise largely homogenous soup.}
}

\end{document}
