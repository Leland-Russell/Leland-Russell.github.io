\documentclass[12pt, twoside]{amsart}    

\usepackage{../current_style}

\begin{document}

\title{Black Bean Soup}
\maketitle
\subtitle{3-4 hours}

\noindent \desc{I make this recipe about 2-3 times a year. I find that I can't make it too often because it is really rich, but when I make it, I'm always happy with it. It's been changed a lot through the years, mostly to get the spiciness right, but if it's too spicy, it can always be toned down with sour cream or some similar topping.}


\recipe{
\textbf{Soup:}
\begin{ingredients}
\item Soup base (with no red pepper flakes)
\item Jalapeño peppers for desired spiciness
\item 1 tablespoon Chili in Adobo, already blended
\item Red wine or red wine vinegar
\item 1 cup dried Black Beans, soaked overnight, not completely cooked
\item smoked paprika
\item 1 teaspoon more of coriander
\end{ingredients}
\textbf{Toppings:}
\begin{ingredients}
\item Onions
\item Lime Juice 
\item Cornbread (see recipe)
\item Cheese
\item Sour Cream
\item Tomatoes
\item Avocadoes
\end{ingredients}
}{
\begin{directions}
\item Follow step 1 from the stew base, including the jalapeño peppers with the vegetables. Before adding garlic add the chili in adobo and allow to thicken on the bare pan for 2-3 minutes before scraping off to the side and proceeding with the garlic. 
\item Before adding the liquid, add the red wine or red wine vinegar (about a cup of red wine or a few tbsp of vinegar) and let it cook down. Then with the liquid, add in paprika, black beans, and coriander.
\item While the soup cooks, finely slice onions and add to a bath of lime juice, water, and salt---enough to submerge about half of the onions---occasionally disturb throughout the rest of the process.
\item Make your favorite cornbread recipe
\item Once the soup has achieved the desired taste and is moderately thick, remove about a third and blend up finely (make sure you don't get the bay leaf), adding back to achieve a thick consistency, blend up more if you want. Add more smoked paprika or chili in adobo to taste.
\item Serve with any of the onions, cornbread, cheese, sour cream, tomato, and Avocado.
\end{directions}
}

\end{document}
