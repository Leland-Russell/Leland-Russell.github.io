\documentclass[12 pt, twoside]{amsart}

\usepackage{changepage}
\usepackage{../current_style}

\begin{document}
\title{Oat-Chocolate Cookies'}
\maketitle
\vspace{-0.75 cm}
\begin{center}
    20 minutes prep, 20 minutes rest, 10 minutes to bake
\end{center}
\vspace{0.375 cm}
Originally, these cookies were made from the apparently somewhat well known \href{https://www.food.com/recipe/neiman-marcus-250-chocolate-chip-cookies-recipe-13307}{\$250 cookie recipe}, but they have developed into something quite different at this point. Now they are just whatever toppings I want to have in cookies. The butter can be easily substituted for coconut oil at which point there's no reason to rest the cookies. Also, it's a very large recipe, it always makes more than I think it will.

\hfill \\
\columnratio{0.4,0.5}
\begin{paracol}{2}	

\large
\noindent \textbf{Ingredients:} \\ 
\normalsize
\begin{itemize}
\item 225 grams / 1 cup / 2 sticks butter
\item 100 grams / 1/2 cup granulated sugar
\item 140 grams / 2/3 cup brown sugar
\item 2 eggs
\item 2 tsp vanilla
\item 225 grams / around 2 1/2 cups oatmeal
\item 250 grams / around 2 cups flour
\item 1/2 tsp salt
\item 1 tsp baking soda
\item 1 tsp baking powder
\item 1/4 cup cocoa powder
\item 2 tsp cardamom
\item 2 tsp nutmeg
\item 2 tsp cinnamon
\item 1/2 tsp ginger
\item 225 grams / 8 oz chocolate chips *
\item 200 grams / 2 cups of chopped nuts
\end{itemize}

\switchcolumn
 \large
\noindent\textbf{Directions}\normalsize\\
 
\begin{enumerate}
\item[1.] Preheat the oven to 375 Farenheit. Cream together butter and sugars. Add egg and vanilla.
\item[2.] For a finer cookie, grind half the oats a little bit in a blender. In a separate bowl, add all dry ingredients up to (and excluding) the chocolate chips.
\item[3.] Add the dry to the wet and add the chocolate and nuts to the batter.
\item[4.] Rest cookies in the fridge for twenty minutes
\item[5.] Cook large cookies for 10 minutes on a greased or parchment-paper-lined baking sheet
\end{enumerate}
\begin{adjustwidth}{20pt}{0pt}
Note: The chocolate chips can really be substituted for anything. I once turned these into raisin cookies by adding a bit more cinnamon and adding a bunch of raisins instead of chocolate. You could add more nuts. You also could make these more like chocolate cookies by laying back on some of the spices and nuts and maybe adding in a little bit more sugar. These cookies can be whatever you want them to be.
\end{adjustwidth}
\end{paracol}
\end{document}
