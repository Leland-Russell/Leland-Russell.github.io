\documentclass[12 pt, twoside]{amsart}

\usepackage{../current_style}

\begin{document}
\title{Cardamom Shortbread ``Buttons''}
\maketitle
\vspace{-0.75 cm}
\begin{center}
    10 minutes to prep, an hour to chill, and 20 minutes to bake.
\end{center}
\vspace{0.375 cm}
These are a newer favorite of mine. They are also quite simple, but take some resting time because they use a butter dough. I've found them to be quite mellow but also pretty addictive. Whenever I make these at school, they take quite a bit longer than 10 minutes to prepare the dough because I don't have a stand mixer, but it should be quite a bit faster with one. Finally, as with most butter dough's, make sure you actually chill the dough. If you want to see what a difference it makes, try making one immediately and then one after chilling, the first one will share many more attributes with a puddle.

\hfill \\
\columnratio{0.4,0.5}
\begin{paracol}{2}	

\large
\noindent \textbf{Ingredients:} \\ 
\normalsize
\begin{itemize}
\item 1/2 (one stick) cup of butter
\item 100g (around a half cup) of sugar
\item 275 (around 2 - 2 1/4 cups) of flour
\item 1 egg
\item 1 tsp baking soda 
\item 1 tsp ground ginger
\item 1 tsp ground cinnamon
\item 2 tsp ground cardamom
\item 1 tsp ground nutmeg
\item 2 tsp vanilla extract
\item 1/4 tsp salt
\end{itemize}

 \switchcolumn
 \large \textbf{Directions} \\
 \normalsize
\begin{enumerate}
\item[1.] Cream butter and sugar in a stand mixer.
\item[2.] Incorporate all the other ingredients. The dough will likely be quite dry, don't be concerned if it's mostly crumbs
\item[3.] Chill dough for an hour, preheat oven to 400 Farenheit.
\item[4a.] With your hands, shape dough into small balls ensuring each is somewhere around 15 g - 25 g, OR
\item[4b.] Roll the dough into a long roll and cut into even pieces, shaping into balls 
\item[5.] Bake for around 10 minutes or until the bottoms are golden on a greased baking sheet or parchment paper. Fridge the remaining dough while these bake.
\item[6.] Let cool for around 10 minutes or until the cookies are solid after coming out of the oven.
\end{enumerate}
\end{paracol}
\end{document}
