\documentclass[12 pt, twoside]{amsart}

\usepackage{../current_style}

\begin{document}
\title{Corn Masa Dumpling Soup}
\maketitle
\subtitle{3-4 hours}

\noindent \desc{I haven't actually made this soup before, but I thought of it one day as a derivative of a soup I had at a restaurant in Maine. They had a light bean soup with coriander. I want to make this, and I haven't yet. Particularly, I want to make sure that the masa dumplings work out well and the coriander has the right proportions. If you are reading this, you must be digging around in my git repo, so if you make it and have any notes, feel free to submit a pull request or something with comments. Or you can contact me at \texttt{lrussell[at]reed[dot]edu}.}

\recipe{
\textbf{Soup:}
\begin{ingredients}
\item 1 Onion 
\item Dried Chili
\item 1 large carrot 
\item 1 bell pepper
\item 1 Acorn Squash or Sweet potato
\item 1-3 teaspoons chili in adobo
\item 3 cloves of garlic
\item Stock or Bouillon 
\item 1 cup of rehydrated dried black beans
\item 1 16oz can of diced tomatoes
\item 3 tsp each of thyme and rosemary
\item 1 Bay leaf
\item 1 tbsp of whole coriander seed
\item Smoked Paprika to taste
\end{ingredients}
\noindent\textbf{Masa Dumplings}
\begin{ingredients}
\item 2 cups corn masa flour 
\item 1 teaspoon salt
\item 1 cup of water
\item 1/4 cup olive oil
\end{ingredients}
}{
\begin{directions}
\item Preheat the oven to 375. While oven preheats, finely dice onion, carrot and bell pepper. Sauté in olive oil for about 20 minutes---until losing structure---with a sprinkling of salt and the chili.
\item While vegetables sauté, cut off (up to a quarter inch) and discard the two ends of the squash. Cut squash in half, removing seeds and membrane, and slice into inch thick annuli. Place squash on lightly oiled foil on a baking sheet and lightly salt and pepper them. Once oven is preheated, bake for around 40 minutes, until almost tender, then set aside.
\item Once vegetables have properly lost their structure and are starting to brown, make room for chili in adobo and sauté for about a minute, keeping it moving. After a few minutes stir the pan to combine everything, finely chop or grate the garlic and do the same, adding a little bit of olive oil to help sauté the garlic.
\item Add beans, a generous amount of liquid along with bouillon, spices, and tomatoes. Let soup simmer, this should go for at least an hour to make the coriander seed softer.
\item While the soup simmers, make the masa balls: Add the dry ingredients to a bowl, add oil, and slowly add water, mixing as you go until the dough forms a Play-Doh consistency. Form into inch diameter hockey puck shaped pieces. 
\item Add squash and Masa balls to the soup. At this point, begin to adjust the soup for consistency. It should be properly watery---an actual soup and not a stew. Add salt, red pepper flakes, smoked paprika, and salt to taste with bouillon or soy sauce.
\end{directions}
}


\end{document}
