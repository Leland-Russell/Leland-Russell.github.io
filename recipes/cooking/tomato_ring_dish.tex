\documentclass[12pt, twoside]{amsart}    

\usepackage{../current_style}

\begin{document}

\title{Tomato Ring Dish}
\maketitle
\vspace{-0.75 cm}
\begin{center}
3-4 hours
\end{center}
\vspace{0.375 cm}
This dish has been evolving for a number of years with my family. Originally, my dad used to make this dish with pine-nuts instead of beans. Then we started adding molasses. I was initially skeptical of the molasses but it adds a really nice depth. After that, my dad started making coconut yogurt and I turned the tomatoes into more of a sauce and it arrived at its current form. One of the beautiful things about this dish is that if made well, all the rings stay intact and provide somewhat distinct flavors. The veggies will be sweet whereas the beans will have a warmer flavor and the tomatoes taste more like summer. If you can, preserve and add the tomato stems or leaves as you might with bay leaves for a more intense tomato flavor.

\hfill \\
\columnratio{0.4, 0.5}
\begin{paracol}{2}
\large
\noindent \textbf{Ingredients:} \\ \normalsize
 \begin{itemize}
\item 2 medium onions
\item 1 Fennel Bulb
\item 1 Large Carrot (and/)or
\item mushrooms (and/)or
\item 1 Bell Pepper 
\item Some Celery 
\item 2 Garlic Cloves
\item 2 Medium/Large Potatoes
\item 1 16oz can of tomatoes
\item 1 tbsp Tomato Paste
\item 1 tsp sugar or Syrup
\item 1-2 jalapeño peppers
\item 1 tbsp basil
\item  2 tsp oregano
\item 1/2 Boullion cube or equivalent
\item 1/4 Cup Olives
\item 2 Cups of Pre-Soaked Beans*
\item 2 tsp Cumin or Coriander 
\item 1-3 Tbsp Molases
\item Wilting greens (i.e.) spinach, broccolli greens, etc. to cover
\item Sour cream/Yogurt
\end{itemize}
\hfill \\

\switchcolumn
\large \textbf{Directions} \\ \normalsize

\begin{itemize}
\item[1.] Chop onions. Place in a large, deep pan with oil over medium heat and lightly salt . While that cooks, chop fennel, carrot, mushrooms, bell peppers, and any other vegetables. Once the onions have cooked a little bit and are releasing juices (5-10 minutes), add other vegetables and more salt and cook for a half hour. All the vegetables should reduce significantly.
\item[2.] Finely chop or grate the garlic. Once the veggies have reduced, make a clearing in the center of the pan for the garlic. Cook with olive oil for about 2 minutes with constant movement. Once they start to brown, incorporate into the rest of the vegetables.
\item[3.] While the vegetables sauté, chop up potatoes and microwave in water until just less than fork tender.
\item[4.] After adding the garlic, form the veggies into a ring on the outside of the pan and add the potatoes. Again keep them moving, but don't worry if they stick a little bit. Cook until the potatoes have color on most sides, around 20 minutes.
\end{itemize}
\end{paracol}
\begin{itemize}
\item[5.] Push the potatoes out to the edge again so you have a second, smaller ring. Add the tomatoes, tomato paste, basil, oregano, bouillon, and olives into the middle. Cook for another 20 minutes or so, until most of the liquid has evaporated.
\item[6.] Rinse the beans thoroughly, when the tomatoes are done, push them to the edge, making a third inner layer and put the beans in the remaining space and topping with cumin or coriander, salt, and black pepper. Lower the heat more and cook for about 10 minutes. Don't worry about cooking them a lot, they should be reasonably distinct from to the rest of the dish.
\item[7.] Drizzle molasses over top and top with greens and cook covered until the greens have wilted. Serve trying to preserve the ringed structure and top with sour cream or something fatty.
\end{itemize}
*Beans can be replaced with any protein, lentils are popular in my family, requiring a little bit of water added and longer cook time. Soy curls could also work well, but should probably be marinated beforehand.

\end{document}